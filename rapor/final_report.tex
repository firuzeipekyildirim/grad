% =========================
% EMU405 Final Report Template (LaTeX)
% Based on "GP FINAL Report format (updated).pdf"
% =========================

\documentclass[12pt]{article}
\usepackage[a4paper,margin=1in]{geometry}
\usepackage{graphicx}
\usepackage{titlesec}
\usepackage{fancyhdr}
\usepackage{pdflscape}
\usepackage{setspace}
\usepackage{tabularx}
\usepackage{xltabular}
\usepackage{amsmath}
\usepackage{parskip}
\usepackage{booktabs}
\usepackage{array}
\usepackage{enumitem}
\usepackage{graphicx}
\newcommand{\sdgicon}[1]{\raisebox{-0.25\height}{\includegraphics[height=4
em]{#1}}\hspace{0.6em}}
\usepackage[table]{xcolor}
\usepackage{array}
\usepackage{colortbl}
\usepackage[T1]{fontenc}
\usepackage[utf8]{inputenc}
\usepackage{booktabs}
\usepackage{cite}
\usepackage{enumitem}

% ---------- Packages ----------
\usepackage[a4paper,margin=1in]{geometry}
\usepackage{setspace}
\usepackage[expansion=false]{microtype}
\usepackage{graphicx}
\usepackage{amsmath,amssymb}
\usepackage[hidelinks]{hyperref}
\usepackage{caption}
\usepackage{booktabs}
\usepackage{longtable}
\usepackage{url}        % \url{} komutu için
\usepackage[hidelinks]{hyperref}  % linkleri düzgün basar (mümkünse en sona yakın)
\usepackage{cite}       % IEEE tarzı [1–3] gibi sıkıştırma/sıralama yapar
\usepackage{cleveref}
% ---------- Formatting ----------
\onehalfspacing
\setlength{\parindent}{0pt}
\setlength{\parskip}{6pt}

% ---------- Title Info (EDIT) ----------
\newcommand{\ProjectTitle}{PROJECT TITLE HERE}
\newcommand{\CourseName}{EMÜ 405 System Analyses and Design I}
\newcommand{\ReportType}{FINAL REPORT}
\newcommand{\AdvisorName}{Advisor Name Here}
\newcommand{\PresentationDate}{Date of Presentation Here}
\newcommand{\CityCountry}{Ankara / TURKEY}


% ---------- Document ----------
\begin{document}

% =========================
% Cover Page
% =========================
\begin{center}

\vspace*{1cm}

\textbf{\Large PLANNING AND OPTIMIZATION OF}\\[0.3cm]
\textbf{\Large CAPACITATED VEHICLE ROUTING WITH}\\[0.3cm]
\textbf{\Large SIMULTANEOUS PICKUP AND DELIVERY IN}\\[0.3cm]
\textbf{\Large LAST-MILE COURIER OPERATIONS}

\vspace{1cm}

\textbf{EMÜ 405 System Analysis and Design I}

\vspace{1cm}

\textbf{FINAL REPORT}

\vspace{1cm}

\textbf{By}

\vspace{1cm}

\begin{tabular}{l l c}
\textbf{ÇAKIR} &  \textbf{IŞIL} & 2230469118 \\
\\
\textbf{ULAŞ} &  \textbf{MUSTAFA ALP} & 2210469034 \\
\\
\textbf{YILDIRIM} &  \textbf{FİRUZE İPEK} & 2220469028 \\
\end{tabular}

\vspace{2cm}

\textbf{Advisor}

\vspace{0.8cm}

\textbf{ERDİ DAŞDEMİR}

\vfill

\textbf{SUBMITTED TO THE DEPARTMENT OF INDUSTRIAL ENGINEERING}\\
\textbf{OF HACETTEPE UNIVERSITY IN PARTIAL FULFILLMENT OF THE}\\
\textbf{REQUIREMENTS FOR THE DEGREE OF BACHELOR OF SCIENCE}

\vspace{1cm}

04.01.2026

\end{center}
\thispagestyle{empty}
\clearpage

% =========================
% Abstract
% =========================
\pagenumbering{roman}

\begin{abstract}
This study addresses a Capacitated Vehicle Routing Problem with Simultaneous Pickup and Delivery (VRPSPD) motivated by last-mile operations. The dataset contains customer stops with geographic coordinates, service day, service type (pickup or delivery), and parcel quantities. We propose a cluster-first, route-second pipeline to enforce operational constraints such as stopcount limits per vehicle and vehicle capacity limits. First, stops are partitioned into geographically compact clusters (interpreted as vehicle tours) using a constrained K-means heuristic that targets balanced cluster sizes. Second, for each cluster, an initial feasible visit order is constructed using a greedy edge-based routing heuristic supported by the Google Distance Matrix API for travel times/distances. To ensure practicality under API limitations, travel queries are batched and candidate pruning is applied. 
A pilot experiment on the first 100 Monday stops demonstrates spatially coherent clusters and produces route-level KPIs (total distance and total duration) for each cluster, enabling comparative evaluation. The full Monday dataset is clustered into 40 balanced tours satisfying the specified stopcount bounds, while large-demand outliers are flagged for split-service handling. The resulting outputs (cluster assignments, route sequences, and KPI summaries) form a reproducible baseline for further optimization and exact/OR-Tools-based refinement.
\end{abstract}


\vspace{0.3cm}
\textbf{Key Words:} [Last-mile logistics; Pickup and delivery operations; Daily route planning; Total distance minimization; Mixed Integer Programming (MIP); Cluster-first Route-second; Gurobi; Google Distance Matrix; Sustainability; Scalability; Decision-support Framework]

\newpage

% =========================
% Contents + Listsp
% =========================
\tableofcontents
\newpage

\listoffigures
\newpage

\listoftables
\newpage

\section*{ABBREVIATIONS}
% Optional but recommended
\begin{itemize}
    \item VRP: Vehicle Routing Problem
    \item VRPSPD: Capacitated Vehicle Routing Problem with Simultaneous Pickup and Delivery
    \item MIP: Mixed-Integer Programming
    \item MTZ: Miller--Tucker--Zemlin
    \item UNSDG: United Nations Sustainable Development Goals
    \item SDG: Sustainable Development Goals
    \item VKT: Vehicle-Kilometers Traveled
    \item KPI: Key Performance Indicator
\end{itemize}

\clearpage

% =========================
% 1. Introduction
% =========================

\pagenumbering{arabic}
\setcounter{page}{0} 
\clearpage
\section{INTRODUCTION}
Last-mile pickup and delivery operations are highly sensitive to fluctuations in daily demand, dense urban structures, and strict operational constraints such as vehicle capacity, driver working hours, and the number of stops that can be made per route. In practice, route plans that do not take these constraints into account may appear efficient on paper but can cause problems in the field, leading to overtime, missed pickups, re-planning during operations, and inconsistent service quality. This project develops a decision support framework that generates operationally feasible daily route plans in the context of parcel delivery, focusing specifically on UPS-like pickup–delivery operations.

The primary output is a daily route plan in executable format: a sequential list of stops is generated for each vehicle in the form of depot → customer stops → depot. In addition to this plan, the following KPIs are reported on a route basis: total distance, total travel time, total service time, number of stops, and capacity/load suitability throughout the tour (including changes in load during the tour). Furthermore, a summary set of indicators is generated at the fleet level to support dispatch decisions; for example, total planned distance/time, average number of stops per route, and capacity utilization distribution.

\subsection{Constraints}
The framework directly considers the fundamental constraints that define courier operations in the field:

\begin{itemize}
    \item \textbf{Vehicle capacity and simultaneous pickup + delivery on the same tour} (dynamic increase and decrease of load throughout the tour)
    \item \textbf{Minimum/maximum number of stops per route} (for operational feasibility)
    \item \textbf{Shift duration limit} (travel + service times must fit within a working day)
    \item \textbf{Zoning/neighborhood constraints where applicable} (to prevent routes from mixing unrealistically distant areas)
    \item \textbf{Basic feasibility rules} such as each stop being served exactly once and each route starting and ending at the depot
\end{itemize}

\subsection{Aim \& Scope}
The primary goal is to minimize total operational costs; in practice, this cost is represented by total travel distance and/or total travel time. If necessary, a secondary term penalizing the number of active vehicles (fleet utilization) can be added to the objective function to prevent the plan from being unnecessarily split into too many vehicles. Therefore, candidate solutions in the report are evaluated using operationally relevant KPIs such as total distance/time, number of routes, compliance of route duration with shift limits, and load/capacity suitability.

Although the framework is designed to be adaptable to different cities, in this project, the application and evaluation are performed using real operational data from Istanbul's Anatolian side. Ankara is mentioned only to highlight the portability and scalability of the approach; Ankara is not an application/test site in this study.

Methodologically, the project uses a Mixed-Integer Programming (MIP) model solved with Gurobi (gurobipy in Python) to produce a “complete solution” at the pilot level for small/medium-scale scenarios; for large daily problems, it uses a cluster-first, route-second (first clustering/assignment, then route generation) approach as a scalable backbone. Route costs are calculated based on actual travel distance/time rather than Euclidean distance (e.g., using a distance matrix similar to Google Distance Matrix); thus, model solutions can be evaluated with operational units.

Section 2 reviews the VRP literature related to pickup–delivery operations and operationally constrained routing. Section 3 clarifies the project scope and assumptions; the elements addressed and those deliberately excluded. Section 4 explains the methodology: the basic VRP MIP template, project-specific extensions (simultaneous pickup–delivery, stop limits, shift limits, zoning), the solution flow with Gurobi, and the backbone of the heuristic approach. Section 5 presents the mathematical model and the existing clustering-based planning line and positions the pilot results within the correct framework. Section 6 discusses limitations and real-world applicability. Section 7 provides conclusions; Section 8 presents future work, including planned extensions such as traffic, multi-depot, advanced metaheuristics, and an updated Gantt chart.

% =========================
% 2. Literature Review
% =========================
\clearpage
\section{LITERATURE REVIEW}

Vehicle Routing Problems (VRPs) constitute a core class of NP-hard combinatorial optimization problems that arise in distribution and service systems. The classical VRP was introduced to formalize the dispatching of vehicles from a depot to multiple customers, and it has since evolved into a large family of variants that better capture operational realities \cite{DantzigRamser1959,Laporte2009,TothVigo2014}. Modern logistics settings, especially last-mile operations, frequently require richer constraints such as capacity, labor regulations, and service heterogeneity \cite{Vidal2019,HalvorsenWeare2014}.

The problem considered in this project aligns with the Vehicle Routing Problem with Simultaneous Pickup and Delivery (VRPSPD), where deliveries and pickups are performed within the same tour and vehicle load changes dynamically during execution. This structure is operationally attractive because it integrates forward and reverse flows, yet it substantially increases routing complexity compared with standard CVRP formulations \cite{TothVigo2014,Vidal2019}. In practice, additional constraints such as stopcount limits, shift duration bounds, and workload balance are often as binding as pure distance minimization, particularly in courier operations \cite{HalvorsenWeare2014}.

Regarding solution approaches, exact Mixed Integer Linear Programming (MIP) models provide optimal or near-optimal solutions for small instances and serve as benchmarks, but scale poorly as the number of customers grows \cite{TothVigo2014,GurobiManual}. Consequently, industrial routing commonly relies on heuristic and metaheuristic methods. Constructive heuristics such as Clarke--Wright savings and sweep-style clustering provide fast initial solutions \cite{ClarkeWright1964,GillettMiller1974}. These solutions are frequently improved using local search operators and metaheuristics (e.g., tabu search, simulated annealing, evolutionary algorithms) to escape local minima and enhance route quality \cite{Gendreau1994Tabu,Kirkpatrick1983,Osman1993SA,Prins2004GA}.

A practically relevant decomposition is the “cluster-first, route-second” paradigm, where customers are first partitioned into vehicle-sized groups and then routed within each group. This idea is closely related to assignment-based heuristics such as Fisher and Jaikumar’s generalized assignment approach \cite{FisherJaikumar1981}. For large-scale daily planning, this two-stage structure supports operational constraints (e.g., bounded number of stops per route) and yields geographically coherent tours, while still allowing subsequent intra-cluster improvement via local search or exact solvers on subproblems.

Finally, recent routing research increasingly emphasizes sustainability, equity, and labor feasibility as first-class objectives rather than side constraints. Pollution-aware routing and sustainable VRP studies show that reductions in vehicle-kilometers traveled can jointly improve cost and environmental outcomes \cite{Shahin2025,Pilati2024,Gouraji2025}. Equity-focused routing is also gaining attention, highlighting that purely cost-driven optimization may systematically disadvantage certain regions unless fairness mechanisms are incorporated \cite{Bonku2024}. These perspectives motivate the inclusion of social, environmental, and regulatory considerations alongside operational KPIs in this project \cite{UN2015}.

Most of the studies we reviewed address VRP-type problems either under ideal assumptions (e.g., simplified distance measurements, limited constraints) or using synthetic data sets that do not fully reflect daily courier operations. In many studies, practical constraints such as the number of stops, fixed shift duration, and zoning—which are critical in the field—take a backseat to achieving good results on standard datasets. This project targets a similar problem class but with a different focus: it addresses a scenario consistent with UPS-like daily planning logic, aims to be scalable to thousands of stops, and places practical constraints such as capacity + pickup–delivery in the same tour + stop number limits + shift suitability at the core of the model. Therefore, our contribution is not to “reinvent” existing methods; it is to combine modeling and heuristic ideas known to work in the literature into a practical decision support framework based on real operational data. In this sense, the work aims to bridge the gap between the academic VRP literature and real courier operations.

% =========================
% 3. Scope of the Project
% =========================
\clearpage
\section{SCOPE OF THE PROJECT}
The scope of this project is the development of a decision-support framework that optimizes
daily pickup and delivery routes for UPS vehicles. The current implementation uses
operational data from the Asian side of Istanbul; however, the framework is fully adaptable
and can be applied to other cities or regions such as Ankara simply by integrating the
relevant operational inputs. This adaptable structure transforms raw data; including stop
coordinates, package volumes and stop-specific attributes, into routing plans that realistically
reflect operational constraints. The project aims to create a mathematical optimization model
capable of generating feasible, cost-efficient, and operationally valid routes. UPS-specific
requirements considered in the scope include vehicle capacity limits, neighborhood limits,
working-hour regulations, and limits on the number of delivery and pickup stops. The
model will also allow vehicles to be assigned to pickup-only or delivery-only routes when
such assignments offer higher efficiency than simultaneous pickup–delivery operations. This
situation may arise in cases involving high-volume customers for example, home-based
“Dolap” (second hand seller web-based interface in Türkiye) sellers or Trendyol retailers who
dispatch large quantities of packages from a single location. Due to scalability challenges in
large problem instances, heuristics and metaheuristics will be used in combination with exact
optimization approaches. The scope also covers the design of controlled experiments to test
the model under different operational scenarios such as changes in demand patterns, route
sizes and capacity levels. Additionally, geographic visualizations of routes will be produced
to support UPS in evaluating operational feasibility and practicality. Certain elements
fall outside the scope of this first phase, including traffic congestion effects, time-window constraints, and multi-depot structures. Weather conditions may be incorporated in later stages, ensuring that the initial phase remains focused on the core routing optimization problem.

\subsection{Problem Definition}
The problem is defined as the optimization of pickup and delivery plans in daily courier
operations. The objective is to assign a set of customer orders each possibly involving both
deliveries and pickups to a vehicle fleet and determine the most efficient sequence of visits
for each route.
To understand the structure of our problem more clearly, we first introduce the classical VRP,which forms the foundation of many modern routing models.
The classical VRP, introduced about sixty years ago by Dantzig and Ramser (1959), and its
many variants have been extensively studied. The VRP seeks to generate a routing plan for a fleet of identical vehicles such that each customer is visited exactly once, each route starts and 12 ends at a depot, and several operational constraints are satisfied. A wide range of heuristic and exact algorithms has been developed for VRP models. For detailed background, the reader may refer to \cite{TothVigo2014},\cite{Laporte2009},\cite{KocLaporte2018},\cite{KocBektaS2016},\cite{Vidal2019}.

When planning the movement of goods or people, Problem-to-Problem Delivery (PDP)
is a variant of the Vehicle Routing Problem that is characterized by pickups and drop-offs that
don’t have to start from a fixed depot, unlike the standard Vehicle Routing Problem. Each
pickup and delivery request in a PDP has one pickup point and one drop-off point, and the
vehicle is required to pick up before dropping off, known as the precedence rule. PDPs come
in three main types, one-to-one, one-to-many-to-one and many-to-many, with the last one
being the most complex \cite{PickupDeliveryReview2024}.
Building on this foundation, we then focus on the specific VRP variants that define the
nature of our project. From an operations research perspective, this problem falls under
the Capacitated Vehicle Routing Problem with Pickup and Delivery (CVRPPD), and more
specifically under the Vehicle Routing Problem with Simultaneous Pickup and Delivery
(VRPSPD), since both operations occur at each stop. These concepts, which we present as
subtopics under the main VRP heading, highlight the additional constraints and dynamics
relevant to our work. In this structure, vehicles must deliver parcels while collecting new
ones during the same route. This causes the vehicle load to change dynamically and requires
it to always remain within capacity limits. This simultaneous structure significantly increases
the complexity of routes compared with classical CVRP models.

\textbf{Problem:}
Design routes that start and end at a UPS depot/warehouse, ensuring that
all required delivery and pickup tasks are completed without violating vehicle capacity
constraints, maximum and minimum workhour duration, or neighborhood adjacency rules.

\textbf{Context:}
UPS Cargo’s daily operations involve assigning thousands of stops across
multiple districts to a dynamically changing fleet. The decision problem is to determine
which vehicle serves which customer, and in which order, such that simultaneous pickup and
delivery operations are feasible throughout the route. Since UPS can hire additional drivers and rent extra vehicles on a daily basis, fleet size does not constitute a restrictive constraint in the model. The real challenges arise from dynamic load changes, district boundaries, heterogeneous stop structures, and ensuring that operational rules remain satisfied at every point along the route. This problem can occur everywhere in the world when a delivery is needed. Well-known spatial, operational and infrastructure challenges in this region could be faced by other national and international UPS hubs, which is why the model that we developed for İstanbul can be adapted and re-employed.

\textbf{Input:}

Customercoordinates, package types (delivery/pickup), number of packages, stop
IDs.

\textbf{Output:}

Visualized route plans, distance and time metrics, required fleet size, number
of cars and walking couriers, ordered stop sequences for each vehicle

The routing problem arises everyday, when managing daily pickup and delivery operations.
Since planning routes is done every morning and the volume of demand can change from
one day to the next, a model that can update and rerun itself is basically a necessity to stay on top of the changing landscape. Because vehicles are locked into strict shift schedules and can’t operate beyond their capacity, poorly planned routes add up to time wasting, excessive kilometers and reduced productivity, which is why it’s so important to have a model that runs regularly.

\subsubsection{UNSDG Consideration for Problem Definition}
\textbf{\sdgicon{E_SDG-goals_icons-individual-rgb-03.png}UNSDG 3 – Promoting Good Health and Well-being}

More realistic and feasible route plans can indirectly benefit occupational health and safety by reducing driver fatigue caused by excessively long trips and time pressure. Additionally, reducing unnecessary vehicle-kilometers can have an impact on reducing local air pollution exposure.

\textbf{\sdgicon{images (1).png}UNSDG 5 – Gender Equality}

More realistic and feasible route plans can indirectly benefit occupational health and safety by reducing driver fatigue caused by excessively long trips and time pressure. Additionally, reducing unnecessary vehicle-kilometers can have an impact on reducing local air pollution exposure.

\textbf{\sdgicon{images.png}UNSDG 8 – Decent Work and Economic Growth}

Directly applying shift duration and operational feasibility constraints creates more predictable and safe daily workloads for couriers.  It has the potential to reduce systematic overtime pressure.

\textbf{\sdgicon{Sustainable_Development_Goal_09Industry.svg.png}UNSDG 9 – Industry, Innovation and Infrastructure}

The study presents a data-driven optimization and decision support approach that transforms operational data into actionable daily plans. The focus will be on strengthening the use of digital infrastructure and analytics in logistics planning.

\textbf{\sdgicon{images (2).png}UNSDG 10 – Reduced Inequalities}

Mere cost minimization may lead to the risk that some regions will be systematically served later or with lower priority; therefore, the planning approach should be interpreted carefully in terms of fair distribution of services.

\textbf{\sdgicon{E_SDG_Icons_11.jpg}UNSDG 11 – Sustainable Cities and Communities}

More efficient and geographically coherent routes can reduce unnecessary vehicle movement in dense urban areas, thereby lowering traffic congestion and the burden on urban operations.

\textbf{\sdgicon{Sustainable_Development_Goal_12ResponsibleConsumption.svg.png}UNSDG 12 – Responsible Production and Consumption}

Reducing unnecessary distance and time supports more efficient use of fuel, vehicle operating time, and labor time while maintaining the same service level. 

\textbf{\sdgicon{images (3).png}UNSDG 13 – Climate Action}

More efficient route planning can contribute to reduced fuel consumption and lower exhaust-related air pollutants by decreasing unnecessary vehicle movement.  This effect becomes particularly pronounced in dense urban areas with heavy stop-and-go trafficore efficient route planning can contribute to reduced fuel consumption and lower exhaust-related air pollutants by decreasing unnecessary vehicle movement.  This effect becomes particularly pronounced in dense urban areas.

\textbf{\sdgicon{E_SDG-goals_icons-individual-rgb-16.png}UNSDG 16 – Peace, Justice, and Strong Institutions}

Clear definition of constraints and the production of plans with measurable KPIs make decision processes more auditable and traceable, thereby strengthening institutional accountability.

\subsection{Objectives of the Study}
\begin{itemize}
    \item Develop an optimization model that generates the most efficient routing plans for pickup and delivery operations, serving as the primary deliverable explicitly requested by UPS.
    \item Increase operational efficiency by structuring routes that reduce unnecessary travel and balance workloads across vehicles.
    \item Enhance customer service performance by reducing delivery times and improving the reliability of daily operations.
    \item Minimize operational costs by reducing the total distance traveled, fuel consumption, the number of hired workers and rented vehicles, and the total work hours required.
    \item Maintain operational feasibility by generating geographically compact routes through assigning stops from adjacent or neighboring neighborhoods.
\end{itemize}

\subsubsection{Real Life Constraints of the Study}
Since UPS can hire additional drivers and rent extra vehicles on a daily basis, the system
operates with a practically unlimited pool of workers and vehicles for modeling purposes.
This means fleet size is not a restrictive constraint in the problem; instead, the primary
limitations arise from route-level factors such as capacity, time, neighborhood adjacency,
and daily demand variations. If routing quality is poor, it can leave couriers with uneven
workloads, a lot of back-and-forth on the road and ultimately, delayed deliveries. Coming
rushing back to the depot isn’t exactly pleasant, and regional managers and performance
analysts are also caught in the mix. Inefficient routes send them spiraling towards higher
operational costs, more fuel usage and longer kilometers traveled. There are around 100 stops to fit in, or roughly one stop every few minutes if we’re counting driving and service time, when planning routes. Coming from one stop to the next, the time taken to park, pick up or drop off packages, sign for things, and so on is pretty short in built up areas, usually just 1-5 minutes per stop. Well-known issues arise when we combine these service times with travel time between them. If a route has too much space between its stops, the driver may end up over the maximum allowed hours. UPS’s guidelines recommend that daily routes shouldn’t exceed about 7.0 to 8.5 hours in order to balance efficiency and the driver’s workload, and if we regularly go beyond that, we’d be looking at overtime costs, or flouting labor agreements. Coming close to the limit is also not advisable, as UPS drivers have a contractual limit of approximately 9.5 hours a day. When a major traffic jam or road closure is encountered, dynamic routing kicks in and sends the driver to a different route, or a company may shift the delivery to another driver, and with real-world factors like car breakdowns, blizzards or drivers getting sick right in the middle of their shift.

\textbf{Primary Constraints}
\begin{itemize}
    \item Stop Count: Minimum 80, Maximum 110 stops per route
    \item Capacity: Max 350 packages (The load must be checked dynamically, deliveries decrease load,pickups increase load)
    \item Duration: Shift time between 7 hours and 8.5 hours
    \item Spatial Clustering: Routes must serve neighboring neighborhoods (no cross-district jumps)
    \item Depot: Start and end at the distribution center
\end{itemize}
In this study, the constraints that define field feasibility are treated as first-class model requirements rather than secondary considerations.

\subsubsection{UPS-Specific Fleet Assumption (Non-binding Fleet Size)}
UPS can hire additional drivers and rent extra vehicles on a daily basis. For modeling purposes, the fleet is treated as practically unlimited, meaning fleet size is not the binding restriction. Instead, feasibility and quality are governed by route-level constraints (capacity, time, stop count, and spatial realism). This assumption is explicitly used to focus the model on generating operationally feasible routes rather than failing due to an artificially tight fleet cap.

\subsubsection{Service Time and Stop Density Context}
Daily plans typically include around 100 stops. Service times in dense urban areas are short (approximately 1--5 minutes per stop), but total route feasibility is dominated by the combination of service time and travel time between stops. If routes are geographically scattered, travel time increases and the plan can exceed allowable working hours, triggering overtime, re-routing, or dispatcher intervention.

\subsection{Scope Boundaries: Addressed vs. Excluded vs. Future Work}

To avoid mixing implemented elements with planned extensions, the project scope is explicitly bounded as follows.

\subsubsection{In Scope (Implemented / Directly Addressed)}
\begin{itemize}
    \item Usage and preprocessing of the Istanbul--Asian side operational dataset.
    \item \textbf{VRP core routing optimization} including capacity feasibility, number-of-stops limits, shift feasibility, depot start/end, and single-service constraints.
    \item Spatial logic for \textbf{neighbourhood adjacency} to maintain spatial realism.
    \item Support for \textbf{pickup-only / delivery-only} assignments for high-volume customers when necessary.
    \item Controlled scenario experiments (e.g., demand patterns, route sizes, capacity levels).
    \item Route visualizations for operational evaluation.
\end{itemize}

\subsubsection{Out of Scope in This Phase (Deliberately Excluded)}
\begin{itemize}
    \item Traffic congestion and time-dependent travel times.
    \item Customer time windows and appointment constraints.
    \item Multi-depot structures and inter-depot allocation decisions.
\end{itemize}

\subsubsection{Planned Extensions (Future Work)}
\begin{itemize}
    \item Weather- and disruption-aware planning (e.g., incidents, closures, breakdowns).
    \item Time-dependent metrics: traffic-aware distance/time matrices.
    \item Generalization to multi-depot and larger-scale operational problems.
    \item More sophisticated heuristic/metaheuristic components for large daily instances.
\end{itemize}

\subsubsection{Ethical Considerations}
The project follows established ethical principles in data use, operational fairness, and social impact.

\textbf{Data Confidentiality}

UPS operational data will be handled securely and used only for academic purposes.
Access is limited to team members involved in modeling and analysis, and no sensitive
information will be shared externally.

\textbf{Fair Workload for Employees}

Routing decisions directly affect couriers’ daily workload. By enforcing constraints
on capacity, shift duration, and stop counts, the model aims to prevent overwork, support
balanced route assignments, and maintain safe working conditions.

\textbf{Impact on Company Operations}

Operational efficiency gains-such as reduced travel distance, fewer rented vehicles, and
lower fuel usagemust not compromise employee well-being or safety. Ethical evaluation
includes balancing cost savings with fair labor practices.

\textbf{Societal and Environmental Considerations}

Optimized routing helps reduce congestion, fuel consumption, and carbon emissions,
contributing to more sustainable last-mile logistics and cleaner urban environments.

\textbf{Team-Level Ethics}

Workload within the project team is distributed according to expertise: modeling, coding,
data analysis, and reporting. Tasks are assigned transparently to ensure fairness and avoid unequal burden. Communication is maintained to adjust workloads when necessary

\subsection{Multidisciplinary Approaches and Interactions}
The project requires a genuinely multidisciplinary collaboration, bringing together students from Industrial Engineering, Data Analytics, Computer Science, and Logistics Management. Each field contributes a critical dimension to a problem that cannot be solved through a single perspective.

Industrial Engineering provides the foundation for systems thinking, mathematical modeling, and optimization, the core tools needed to structure the routing problem rigorously. Data Analytics enables the transformation of raw operational data into actionable insights, ensuring that demand patterns, customer densities, and service behaviors are properly understood before modeling. Computer Science contributes algorithmic design, computational efficiency, and the implementation of exact and heuristic methods, which are essential for solving large-scale instances that exceed traditional optimization limits. Logistics Management brings operational realism, structuring the solution within UPS’s operational constraints, behaviors, and real-world uncertainties of last-mile courier operations.

This rich combination of disciplines is not an academic luxury but a necessity. Last-mile routing is a hybrid challenge mathematical, operational, technological, and organizational at the same time. Only an integrated, multidisciplinary approach can capture the complexity of simultaneous pickup–delivery operations, dynamic loads, neighborhood adjacency, and day-to-day feasibility \cite{Jazemi2023},\cite{Shuaibu2025}.

Throughout the project, the team will also engage with UPS personnel from operations, planning, and IT departments. These interactions ensure that the theoretical models align with actual on-ground practices, data structures, and operational constraints. By bridging academic expertise with industry experience, the project does more than build an optimization model; it creates a realistic, end-to-end decision-support framework that reflects the interdisciplinary nature of modern logistics systems \cite{Zhao2024}.
% =========================
% 4. Methodologies
% =========================
\clearpage
\section{METHODOLOGIES}

In this study, a hybrid methodological perspective is adopted to address the Pickup and Delivery Vehicle Routing Problem under capacity, time, and neighborhood-related constraints. A Mixed Integer Programming (MIP) model is formulated to provide an exact and formal representation of the problem structure, which is intended to be used for benchmarking and validation in later stages of the study.

Given the scale and operational nature of real-world routing instances, preliminary solution insights are obtained through a heuristic cluster-first, route-second approach. Rather than employing a wide range of heuristic or metaheuristic techniques, the methodology focuses on a single, clearly defined heuristic structure that aligns with the underlying VRP formulation and supports scalability.

This methodological choice reflects common practices in VRPSPD and CVRPPD research, where exact models are used for structural analysis and benchmarking, while heuristic approaches are employed to generate feasible solutions for larger instances.

\subsection{Classical VRP as the Baseline Optimization Model}

The Vehicle Routing Problem (VRP) can be expressed through a standard Mixed Integer Programming (MIP) formulation that serves as the foundation for many practical routing models. In its classical form, the objective is to determine a set of vehicle routes that start and end at a depot, serve all customers exactly once, and minimize total operational cost under capacity and time limitations.

\subsubsection{Parameters}
\begin{itemize}
    \item $d_{ij}$: travel distance between nodes $i$ and $j$
    \item $t_{ij}$: travel time between nodes $i$ and $j$
    \item $s_i$: service time at customer $i$
    \item $Q$: vehicle capacity
    \item $T^{\max}$: maximum route duration (shift length)
\end{itemize}

\subsubsection{Decision Variables}
\begin{itemize}
    \item $x_{ijk} \in \{0,1\}$: equals 1 if vehicle $k$ travels directly from node $i$ to node $j$
    \item $z_k \in \{0,1\}$: equals 1 if vehicle $k$ is used
    \item $u_{ik}$: sequencing variable for subtour elimination (MTZ)
\end{itemize}

\subsubsection{Objective Function}
The classical VRP objective minimizes total travel cost:
\begin{equation}
\min \sum_{k} \sum_{i} \sum_{j} d_{ij} x_{ijk}
\end{equation}

Optionally, a fixed cost can be added to penalize excessive vehicle usage:
\begin{equation}
\min \sum_{k} \sum_{i} \sum_{j} d_{ij} x_{ijk} + \sum_k F z_k
\end{equation}

\subsubsection{Core Constraints}
\begin{itemize}
    \item \textbf{Flow conservation:} each customer has exactly one incoming and one outgoing arc
    \item \textbf{Depot constraints:} each active vehicle starts and ends at the depot
    \item \textbf{Capacity constraints:} vehicle load does not exceed $Q$
    \item \textbf{Shift duration constraints:} total travel and service time does not exceed $T^{\max}$
    \item \textbf{Subtour elimination:} MTZ constraints ensure route connectivity
\end{itemize}

This formulation serves as a baseline framework for more realistic routing models.\cite{TothVigo2014} \cite{laporte1992vrp}

\subsection{Extensions to the Classical VRP Baseline}

The problem addressed in this study extends the classical VRP to reflect real-world courier operations. The main differences are summarized as follows:

\begin{itemize}
    \item \textbf{Simultaneous Pickup and Delivery:} Each customer may require both pickup and delivery, resulting in dynamic vehicle loads throughout the route.
    \item \textbf{Stop Count Constraints:} Each route must include a minimum and maximum number of stops to ensure workload balance.
    \item \textbf{Shift Duration Limits:} Routes must comply with strict working-hour regulations.
    \item \textbf{Neighborhood Constraints:} Routes are restricted to geographically coherent areas to prevent unrealistic cross-district routing.
\end{itemize}

These extensions transform the problem into a \textbf{Capacitated Vehicle Routing Problem with Simultaneous Pickup and Delivery (VRPSPD)}. \cite{min1989vrpspd}

\subsection{Exact Optimization via Gurobi}

In order to provide an exact optimization reference for the extended VRP formulation, a Mixed Integer Programming (MIP) approach is considered to be solved using Gurobi. The model is planned to be implemented in Python through the \texttt{gurobipy} interface. Rather than focusing on solver specific details, this section describes how Gurobi is intended to be used within the overall solution pipeline.

\subsubsection{Planned Gurobi-Based Solution Workflow}

The planned implementation follows a sequential workflow that transforms raw input data into executable vehicle routes:

\begin{enumerate}
    \item \textbf{Data Reading:}  
    Customer and depot data are read using Python from structured input files containing geographic coordinates, delivery quantities, pickup quantities, and service times.

    \item \textbf{Model Construction:}  
    The extended VRP formulation is encoded in Gurobi by defining decision variables, the objective function, and all constraints using the \texttt{gurobipy} modeling interface. This step translates the mathematical formulation into a solver-readable model.

    \item \textbf{Optimization:}  
    The resulting MIP model is solved using Gurobi’s branch-and-bound algorithm. The solver searches for an optimal or near-optimal solution within specified computational limits.

    \item \textbf{Route Reconstruction:}  
    After optimization, the values of the decision variables are extracted and postprocessed in Python to reconstruct vehicle routes. Each route is represented as an ordered sequence of customer visits starting and ending at the depot.
\end{enumerate}

This workflow explicitly defines how Gurobi integrates into the proposed methodology and how optimization outputs are converted into operational routing plans.

\subsection{Planned Heuristic Solution Approach}

In this study, a \textbf{Cluster-First, Route-Second Heuristic} is selected as the primary solution approach. At the current stage of the project, this heuristic has not yet been implemented; however, its structure and rationale are clearly defined to guide the subsequent implementation phase.

The selection of this heuristic is motivated by its strong alignment with the structure of the classical VRP and its ability to naturally incorporate the operational constraints introduced in Section~4.2. 

\subsection{Rationale for Heuristic Selection}

The primary motivation for selecting a cluster-first, route-second heuristic is scalability. Exact optimization methods may become computationally impractical for large-scale daily routing instances. By decomposing the problem into assignment and routing phases, the heuristic reduces computational complexity while preserving feasibility.

Additionally, the assignment-first logic allows capacity limits, pickup and delivery balance, stopcount bounds, and shift duration constraints to be enforced before routing decisions are made.

\subsection{Planned Heuristic Workflow}

The planned heuristic follows a structured workflow that replaces global optimization with problemdriven constructive decisions. The intended steps of the heuristic are as follows:

\begin{enumerate}
    \item \textbf{Input Data Processing:}  
    Customer and depot data will be read, including geographic locations, pickup and delivery quantities, and service times.

    \item \textbf{Clustering (Assignment) Phase:}  
    Customers will be assigned to vehicle-sized clusters based on geographic proximity and operational constraints such as vehicle capacity, stopcount limits, pickup-delivery balance, and shift duration.

    \item \textbf{Route Construction Phase:}  
    For each cluster, a feasible visit sequence will be constructed, ensuring that each route starts and ends at the depot.

    \item \textbf{Feasibility Evaluation:}  
    The constructed routes will be evaluated against all operational constraints, and adjustments will be planned if violations occur.

    \item \textbf{Final Route Output:}  
    The final output will consist of executable vehicle routes represented as ordered customer visit sequences.
\end{enumerate}
\cite{FisherJaikumar1981}
\subsection{Algorithmic Logic and Conceptual Foundation}

The heuristic is conceptually aligned with the assignment-first routing framework proposed by Fisher and Jaikumar (1981). In this framework, the VRP is decomposed into an assignment problem followed by a routing problem within each assignment.

Rather than relying on metaheuristic search, the proposed heuristic emphasizes feasibility-driven construction, where operational constraints guide customer assignment and route formation from the outset.

The current clustering procedure is not an implementation of the Fisher and Jaikumar algorithm. A simpler distance-based clustering is used for initial analysis, while the Fisher and Jaikumar approach is planned for future stages of the study.
\cite{FisherJaikumar1981}
\subsection{Application to Real UPS Operational Data}

At last, the methods will be put into practice with the real UPS data from the Asian Side of Istanbul. The effective validation of the following will be the objective: operational feasibility, geographic compactness, accuracy when compared to existing manual routes, and distance, cost, and workload distribution being the areas where improvements are potentially made. This last step is a proof of the proposed approach being applicable and its value in real-life courier operations.[35-37]

\subsection{Limitations of the Proposed Methodologies}

Every single optimization approach involves the selection of a proper path with certain trade-offs between the solution quality and computational efficiency. Here are the potential limitations of the chosen methodologies for this study:

\subsubsection{Limitations of Exact Optimization (MIP)}

\textbf{Computational Complexity (NP-Hardness)}: The main drawback of the MIP model is its inability to scale. The Vehicle Routing Problem belongs to the NP-hard category. As the number of customer nodes grows, the time taken to fetch the global optimal solution increases at an exponential rate. For our dataset of 80--110 stops per truck, getting the MIP solved to optimality in a reasonable amount of time (say, 1 hour) is mathematically impossible with the existing computing power.

\textbf{Rigidity}: Exact models necessitate exact linear constraints. Using highly complex, non-linear real-world conditions (like dynamic traffic speeds that vary every minute) can make the model unsolvable.

\subsubsection{Limitations of Heuristic Approaches (Fisher \& Jaikumar)}

\textbf{Optimality Gap}: Heuristics, in contrast to MIP, do not offer any guarantee of reaching the global optimal solution (the very best route). They proceed to look for ``good enough'' (near-optimal) solutions. The possibility of being confined to a local maximum where the algorithm believes it has found the best route but actually overlooks a better one is always present.

\textbf{Sensitivity to Initialization}: The efficacy of the ``Cluster-First, Route-Second'' method, especially the Fisher and Jaikumar algorithm, is greatly dependent on the different initial selections of ``seed points'' (centers of the clusters). If the seed points are not appropriately selected, then the resulting clusters may have uneven distribution or be geometrically inefficient which will be hard to fix in the subsequent routing phase.

\textbf{Parameter Tuning}: It’s the case with heuristics that they sometimes necessitate the really careful tuning of their parameters. Misconfigured parameters could result in either bad convergence or long runs without any enhancement in the solution quality at all.

\clearpage
\section{ANALYSIS AND RESULTS}

\subsection{Analysis}
\subsubsection{Data}
To undertand our data and to be able to work with it we first cleaned the names of the columns and replaced all Turkish words to English. Here is a quick look at our data. 
\begin{table}[h]
\centering
\caption{Stop-count statistics for the full Monday dataset and the pilot subset.}
\label{tab:dataset_summary}
\resizebox{\textwidth}{!}{%
    \begin{tabular}{l ccc ccc}
    \toprule
    \textbf{Dataset} & \textbf{Stops (N)} & \textbf{Clusters} & \textbf{Min Stops} & \textbf{Max Stops} & \textbf{Mean} & \textbf{Std. Dev.} \\
    \midrule
    Monday dataset (full) & 3798 & 40 & 80 & 110 & 94.95 & 14.88 \\
    Pilot subset (first 100) & 100 & 5 & 5 & 30 & 20.00 & 9.67 \\
    \bottomrule
    \end{tabular}%
}
\end{table}

The dataset consists of a depot and a set of customers with known locations, delivery demands, and pickup demands. Vehicle capacity constraints and distance-based travel costs are incorporated to model a realistic VRPSPD setting.

\subsubsection{Mathematical Model}

\textbf{Sets and Indices}

\begin{itemize}
    \item $N = \{0,1,\dots,n\}$: Set of all nodes, where node $0$ denotes the depot.
    \item $C = N \setminus \{0\}$: Set of customer nodes.
    \item $K$: Set of available vehicles.
\end{itemize}

\textbf{Parameters}

\begin{itemize}
    \item $d_{ij}$: Distance between nodes $i$ and $j$.
    \item $t_{ij}$: Travel time between nodes $i$ and $j$.
    \item $s_i$: Service time at node $i$.
    \item $D_i$: Delivery demand of customer $i$.
    \item $P_i$: Pickup demand of customer $i$.
    \item $Q$: Vehicle capacity.
    \item $F$: Fixed cost of using a vehicle.
    \item $E_i$: Binary parameter equal to 1 if customer $i$ is an express delivery.
    \item $\alpha$: Penalty coefficient representing express priority.
    \item $\beta$: Backlog (delay) penalty coefficient, where $\beta \gg \alpha$.
    \item $M$: A sufficiently large constant.
\end{itemize}

\textbf{Decision Variables}

\begin{itemize}
    \item $x_{ijk} \in \{0,1\}$: Equals 1 if vehicle $k$ travels from node $i$ to node $j$.
    \item $y_{ik} \in \{0,1\}$: Equals 1 if customer $i$ is served by vehicle $k$.
    \item $z_k \in \{0,1\}$: Equals 1 if vehicle $k$ is activated.
    \item $q_{ik} \ge 0$: Load of vehicle $k$ after visiting node $i$.
    \item $t_{ik} \ge 0$: Arrival time of vehicle $k$ at node $i$.
    \item $u_{ik} \ge 0$: MTZ sequencing variable.
\end{itemize}

\textbf{Objective Function}

\begin{equation}
\min
\sum_{k \in K} \sum_{i \in N} \sum_{j \in N} d_{ij} x_{ijk}
+ \sum_{k \in K} F z_k
+ \sum_{i \in C}
\left[
\alpha E_i
+ \beta \left(1 - \sum_{k \in K} y_{ik}\right)
\right]
\end{equation}

\textbf{Constraints}

\paragraph{Customer Assignment}
\begin{equation}
\sum_{k \in K} y_{ik} \le 1 \quad \forall i \in C
\end{equation}

\paragraph{Vehicle Activation}
\begin{equation}
y_{ik} \le z_k \quad \forall i \in C, k \in K
\end{equation}

\paragraph{Flow Conservation}
\begin{equation}
\sum_{j \in N} x_{ijk} = y_{ik}
\quad \forall i \in C, k \in K
\end{equation}

\begin{equation}
\sum_{i \in N} x_{ijk} = y_{jk}
\quad \forall j \in C, k \in K
\end{equation}


\paragraph{Depot Constraints}
\begin{equation}
\sum_{j \in C} x_{0jk} = z_k
\quad \forall k \in K
\end{equation}

\begin{equation}
\sum_{i \in C} x_{i0k} = z_k
\quad \forall k \in K
\end{equation}


\paragraph{Stop Count Constraints}
\begin{equation}
80 z_k \le \sum_{i \in C} y_{ik} \le 110 z_k
\quad \forall k \in K
\end{equation}


\paragraph{Dynamic Capacity Constraints}
\begin{equation}
q_{jk} \ge q_{ik} - D_j + P_j - M(1 - x_{ijk})
\quad \forall i,j \in N, k \in K
\end{equation}

\begin{equation}
0 \le q_{ik} \le Q
\quad \forall i \in N, k \in K
\end{equation}


\paragraph{Time Feasibility}
\begin{equation}
t_{jk} \ge t_{ik} + s_i + t_{ij} - M(1 - x_{ijk})
\quad \forall i,j \in N, k \in K
\end{equation}


\paragraph{MTZ Subtour Elimination}
\begin{equation}
u_{ik} - u_{jk} + |C| x_{ijk} \le |C| - 1
\quad \forall i \neq j \in C, k \in K
\end{equation}

\begin{equation}
1 \le u_{ik} \le |C|
\quad \forall i \in C, k \in K
\end{equation}

The proposed optimization model is formulated as a \textbf{Capacitated Vehicle Routing Problem
with Simultaneous Pickup and Delivery (VRPSPD)} and incorporates a priority-based penalty
mechanism. The model is used as the analytical foundation to evaluate the impact of
delivery priorities on routing decisions, fleet utilization, and service feasibility.

\textbf{(1) The objective function} minimizes the total operational cost, which consists of travel distance cost, fixed vehicle usage cost, and prioritized penalty costs.
The first term represents the total distance traveled by all vehicles.
The second term penalizes the use of additional vehicles, encouraging fleet efficiency.
The third term introduces a prioritized penalty structure, where express customers
($E_i = 1$) incur an additional priority cost $\alpha$, while customers that remain
unserved within the planning horizon incur a backlog penalty $\beta$, with $\beta \gg \alpha$.
This structure ensures the following priority order:
\[
\text{delayed express} > \text{express} > \text{delayed normal} > \text{normal}.
\]
This structure directly affects routing decisions by prioritizing express
and delayed deliveries under capacity and operational constraints.

\textbf{(2) Customer Assignment Constraint:} Only one vehicle can be used in attending to each customer to prevent duplication.

\textbf{(3) Vehicle Activation Constraint:} The vehicle comes into effect as a serves one customer vehicle, thus determining that fixed costs are only charged based on the utilization of the vehicles..

\textbf{(4) (5) Flow Conservation Constraints:} These constraints ensure route continuity by guaranteeing only one incoming and one outgoing arc to all visited customers.

\textbf{(6) (7) Depot Constraints:} Any given vehicle that starts must start from the depot and return after finishing his duty.

\textbf{(8) Stop Count Constraint:} These constraints limit the number of customers served by each vehicle to a predefined operational interval

\textbf{(9) (10) Dynamic Capacity Constraints:} These constraints ensure that vehicle loads dynamically decrease due to deliveries and increase due to pickups, while always remaining within capacity limits.

\textbf{(11) Time Feasibility Constraint:} These constraints ensure temporal feasibility by accounting for travel times and service durations along each route.

\textbf{(12) (13) MTZ Subtour Elimination Constraints:} The Miller–Tucker–Zemlin (MTZ) constraints prevent the formation of subtours disconnected from the depot, ensuring a single continuous route for each vehicle.

At the current stage of the study, the mixed-integer programming model presented above has been formulated but \textbf{not yet solved}. The primary role of this model is to serve as a formal and exact representation of the problem structure, which will later be used for benchmarking and validation purposes.

While solving the full MIP model is planned for subsequent stages using Gurobi, preliminary routing insights are obtained through a simplified cluster-first, route-second approach.

\subsubsection{Data pipeline and validation}
The operational dataset contains stop identifiers, geographic coordinates (longitude/latitude), service day, service type (delivery or pickup), and parcel quantities. Prior to optimization, the data are filtered by service day and basic consistency checks are applied (missing coordinates, invalid quantities, and duplicate stop IDs). For the Monday dataset used in clustering, a total of 3{,}798 stops are processed, consisting of 3{,}102 delivery stops and 696 pickup stops, with an aggregate parcel count of 14{,}826.

A key feasibility consideration is the vehicle capacity constraint ($Q=350$ packages). While the full dynamic-load feasibility requires the ordered sequence and initial load assumptions, the raw stop-level quantities still provide an important screening mechanism. In the Monday data, 9 stops exceed 350 packages; these customers are flagged as high-volume outliers that may require split-service, dedicated vehicles, or preassignment rules as seen in appendix D, \cref{tab:qty_outliers}

\subsubsection{Cluster-first stage (balanced geographic clustering)}
To obtain operationally manageable tours, stops are partitioned into geographically compact clusters that represent candidate vehicle routes. The clustering stage is designed to satisfy UPS-style stopcount bounds by construction, targeting 80--110 stops per route. On the full Monday dataset, the procedure produces 40 clusters with sizes ranging from 80 to 110 (mean 94.95, standard deviation 10.06), indicating that the size-balancing mechanism successfully controls workload distribution at the route level (Appendix Table~\cref{tab:dataset_summary}).

For reporting and debugging, a pilot experiment is conducted on the first 100 Monday stops. This pilot instance is clustered into 5 groups, which allows visualization and end-to-end KPI computation (Figure~\cref{fig:cluster_first100}).

\subsubsection{Route-second stage (intra-cluster sequencing and travel metrics)}
For each cluster, a feasible visit order is constructed using a greedy edge-based heuristic that iteratively extends the route by selecting the next promising stop under the chosen travel metric. Pairwise travel distances and durations are obtained via the Google Distance Matrix API under driving mode \cite{GoogleDistanceMatrixAPI}. Due to API quota limitations, queries are batched and cached, and the pilot reporting is based on non-traffic durations; traffic-inclusive durations are reported separately as reference values (Appendix Tables~\cref{tab:sample10_dur_min} and \cref{tab:sample10_traf_min}).

Route-level KPIs are computed as the sum of arc costs along the ordered sequence, including depot departure and return. The primary KPIs reported are total distance (km), total travel duration (min), and derived efficiency ratios such as km per stop and minutes per stop.


\subsection{Results}

\subsubsection{Clustering Results}

The analysis presented in this section is based on exploratory trials rather than final or optimized solutions. At this stage, customer clustering is performed using the k-means algorithm as a simple, distance-based approach in order to gain preliminary insights into the spatial structure of the problem. Clustering is carried out solely based on geographic proximity, without explicitly incorporating operational constraints. The number of clusters is predefined and set equal to the number of available vehicles, with each cluster interpreted as a candidate service region for a single route in the subsequent routing phase. Constraints such as stop-count limits, vehicle capacity, and shift duration are not enforced as hard constraints during clustering; instead, they are considered indirectly by monitoring cluster sizes and aggregate demand levels and are evaluated during route construction. The applied clustering approach should not be interpreted as an implementation of the Fisher--Jaikumar assignment heuristic. While the Fisher--Jaikumar methodology provides a more structured and capacity-aware assignment framework, it is reserved for future stages of this study. Accordingly, the results reported in this section are indicative and exploratory in nature, and are intended to support methodological understanding rather than to represent final optimized outcomes.

The pilot experiment on the first 100 Monday stops yields 5 feasible clusters and corresponding route KPIs (Table~\cref{tab:pilot_kpis_first100}). The results show meaningful variation across clusters. Clusters 0, 1, and 3 represent dense urban tours with 17--30 stops and total travel distances around 107--121 km, while cluster 4 is comparatively shorter (81.91 km). In contrast, cluster 2 contains only 5 stops yet results in the largest distance (130.19 km), indicating a geographically distant or spatially sparse set of customers that dominates route cost despite the small stop count. This pattern is operationally important because it highlights that balanced stopcount constraints alone do not guarantee balanced travel burden; additional balancing criteria or geographic constraints may be required to avoid creating “few-stop but long-drive” routes.

At the KPI level, the observed km-per-stop and minutes-per-stop ratios provide a compact measure of route compactness. For example, cluster 2 exhibits the highest km-per-stop and minutes-per-stop values, confirming that it is the least compact tour in the pilot. Figures~\cref{fig:kpi_km_first100} and \cref{fig:kpi_min_first100} visualize these differences across clusters. An example ordered route sequence (Monday cluster 0) is reported in Appendix Table~\cref{tab:route_seq_cluster0} to illustrate the output format used by the decision-support pipeline.

Overall, the pilot confirms that the cluster-first, route-second approach produces spatially coherent route candidates and generates reproducible, interpretable KPIs that can be used for comparison, debugging, and subsequent improvement (e.g., 2-opt refinement or exact optimization within clusters).
\begin{table}[h]
\centering
\caption{Pilot Monday routing KPIs for the first 100 stops.}
\label{tab:pilot_kpis_first100}
% \resizebox{genişlik}{yükseklik}{içerik}
\resizebox{\textwidth}{!}{%
    \begin{tabular}{l ccc ccc cc}
    \toprule
    \textbf{Cluster} & \textbf{Stops} & \textbf{Del.} & \textbf{Pick.} & \textbf{Pkgs} & \textbf{km} & \textbf{min} & \textbf{km/st} & \textbf{min/st} \\
    \midrule
    0 & 17 & 15 & 2 & 152 & 107.37 & 163.60 & 6.32 & 9.62 \\
    1 & 30 & 28 & 2 & 58 & 111.10 & 210.33 & 3.70 & 7.01 \\
    2 & 5 & 5 & 0 & 7 & 130.19 & 115.18 & 26.04 & 23.04 \\
    3 & 26 & 26 & 0 & 45 & 120.84 & 203.92 & 4.65 & 7.84 \\
    4 & 22 & 22 & 0 & 105 & 81.91 & 149.05 & 3.72 & 6.78 \\
    \bottomrule
    \end{tabular}%
}
\end{table}

The performance metrics for the initial routing phase are summarized in \cref{tab:pilot_kpis_first100}. The data presented here reveals specific operational challenges arising from the standard clustering approach. Specifically, while the $k$-means algorithm successfully groups locations based on spatial density, it does not inherently account for the required operational constraints, such as the target workload balance of 80 to 110 stops per route. A notable example is observed in Cluster 2, where the high geographical dispersion results in 130.19 km of travel for only 5 stops, indicating a need for heuristic modifications in the clustering logic to ensure efficiency.

Furthermore, it should be noted that the distance and travel time calculations were performed using the Google Distance Matrix API \cite{GoogleDistanceMatrixAPI}. Due to the daily quota limits of the free-tier service, the current analysis is conducted on this representative pilot dataset of the first 100 stops. These results serve as a preliminary validation of the digital optimization model, aligning with SDG 9.4 by aiming to increase resource efficiency through technological integration. Future work will focus on scaling these optimizations to the full dataset by incorporating modified clustering constraints and extended API resources.


\begin{figure}[h]
\centering
\includegraphics[width=0.85\textwidth]{assets/figs/cluster_plot_first100.png}
\caption{Cluster distribution for the first 100 Monday stops.}
\label{fig:cluster_first100}
\end{figure}

As seen in \cref{fig:cluster_first100}, the green points representing Cluster 2 form an isolated group. This illustrates a typical characteristic of the $k$-means algorithm, which prioritizes spatial density over operational parameters when performing automatic clustering without predefined stop limitations. While Cluster 2 is geographically distinct, its low stop count (5 stops) and high distance-to-stop ratio lead to the inefficiencies observed in \cref{tab:pilot_kpis_first100}. 

These findings align with SDG 9.4, as identifying such digital optimization gaps is the first step toward developing more resource-efficient infrastructure. Addressing these clustering anomalies through heuristic modifications will be essential to ensure that each route maintains an optimal workload, thereby reducing unnecessary vehicle wear and fuel consumption.

\subsubsection{Expected Outcomes and Model Capabilities}

While the suggested model has not been solved by computation, its structure provides an actual guess of the output and insights that could be obtained after solving it with any solver, like Gurobi.

The solution will likely provide, after its determination, perfect routes for each active vehicle, detailing the sequence through which customers are visited by the depot and vice versa. Additionally, the model will provide the minimum number of vehicles and the distribution of customers to the vehicles according to the operating and capacity limits.

The expected quantitative outputs of the model are total traveled distance, number of activated vehicles, vehicle-level load profiles, and the number of served and unserved customers. These outputs enable an evaluation of transportation efficiency and service feasibility under different demand compositions.

Due to the embedded priority-based penalty structure, the model is expected to clearly
differentiate between express and normal deliveries. Express customers are anticipated
to be served with higher priority, potentially resulting in longer routes or additional
vehicle deployment when necessary to avoid high backlog penalties.

From a practical standpoint, the expected outputs of the model can support UPS decision
makers in evaluating trade-offs between service quality and operational cost. By
adjusting the penalty parameters, managers can explore alternative operational
strategies and assess the cost implications of stricter or more flexible service
priorities before implementation.

% Example figure block:
% \begin{figure}[h]
% \centering
% \includegraphics[width=0.85\textwidth]{example_figure.png}
% \caption{Example Figure Caption (replace with your figure)}
% \label{fig:example}
% \end{figure}

% =========================
% 6. Discussions
% =========================
\clearpage
\section{DISCUSSIONS}
% Report all sub-sections and relate relevant UNSDGs where appropriate

\subsection{Study limitations}
There are three key practical limitations that restrict how closely our findings align with real-world operations. 

First, our model simplifies travel times and constraints; however, in actual logistics networks, travel times vary sharply due to traffic incidents, road closures, weather conditions, and time-dependent congestion, making it impossible to accurately measure these situations with a single static prediction. 

Second, real-time travel data necessitates quota-limited and costly external map APIs (e.g., Google Maps); when the number of stops is high and repeated queries are required for multiple scenarios, the cost of frequently updating large-scale origin-destination matrices increases. Therefore, fully dynamic re-optimization at fine time resolutions becomes financially and computationally infeasible for many teams or small-scale deployments. 

Finally, as the sample size increases, the routing problem and binary distance evaluations grow rapidly, complicating both optimization and data acquisition demands. Consequently, the results provided should be interpreted as a robust foundation under controlled assumptions, not a perfect replica of live operations.

\subsection{Applicability of the project under real life situations}
Despite current limitations, the proposed workflow can be effectively implemented as a decision support tool in real life: Companies typically plan routes using historical averages/hourly traffic profiles and limited API data rather than simulating everything instantly; they only update in cases of significant delays, reassignments, or new pickups. Realistic implementation of the method involves daily or shift-based batch planning and using paid map APIs only for targeted verification, model calibration, or updating high-uncertainty connections; thus, route quality is maintained while reducing dependence on expensive real-time queries. Organizations with cost constraints can switch to OpenStreetMap-based open routing tools instead of paid APIs, or develop internal travel time estimators from historical GPS tracks to achieve a more sustainable solution at the cost of some accuracy. Ultimately, the project is most suitable for scenarios focused on enhancing operational feasibility and route quality under budget constraints; it must be acknowledged that full real-time and fully accurate simulation is not always technically or financially rational.

\subsection{Health and Safety Issues}
The proposed decision-support framework aims to directly impact the occupational health and public safety risks related to large-scale pickup and delivery operations. Traditional routing approaches are characterized by fragmented planning and static definitions of routes, which result in drivers' excessively long working hours, unsafe driving behaviors, and last-minute adjustments of the operations. Such conditions increase accident probability, driver fatigue, and exposure to physical strain during repeated loading and unloading activities.

\subsubsection{Occupational Health Risks}

Manual routing systems often lead to drivers having imbalanced workloads. Previous research in labor-constrained VRP shows that long route duration and high stop density are critically linked with both fatigue-related accidents and musculoskeletal injuries of drivers. Explicit consideration of this aspect, by setting strict shift duration-thus, at least 7 hours and at most 8.5 hours-and capping the number of daily stops per vehicle between 80 and 110, embeds explicit upper bounds within. The algorithm then works as a preventable health instrument rather than a reactive safety measure.

\subsubsection{Public Health \& Safety}

From a broader public safety perspective, optimized routing reduces total vehicle-kilometers traveled (VKT). Pollution Routing Problem (PRP) literature indicates that even modest reductions in urban delivery mileage translate into measurable improvements in air quality by lowering emissions of NOx, PM, and CO$_2$, which directly impacts respiratory health outcomes.

\subsubsection{Relevant UNSDGs:}

\begin{itemize}[leftmargin=1.5cm]
    \item \textbf{SDG 3 -- Good Health and Well-Being:} Routing optimization is relevant to goal 3 through the occupational health pathway. Using the occupational health perspective, the study of the VRP when the number of workers is the constraint shows that overly long routes and the number of stops have been observed to have a strong positive relationship to the incidence of fatigue accidents and musculoskeletal injuries among drivers [10].

    \item \textbf{SDG 8 -- Decent Work and Economic Growth:} Decent work in logistics is tightly linked to predictability, workload balance, and regulatory compliance. Halvorsen-Weare and Savelsbergh~[10] show that ignoring labor constraints in routing leads to systematic overwork and informal overtime practices. 
\end{itemize}

\subsection{Legal Issues}
Legal issues arising out of last-mile courier operations encompass moral concerns and go farther because:
Firstly, the two contrasting approaches to the issue are rooted in compliance. The activities of the
are dependent on binding labor legislation, working safety standards, and contract
obligations. Any violation, however small, leads to companies facing administrative fines or lawsuits. Most companies use
disputes, and reputational damage

\subsubsection{Regulatory Constraints and Compliance Risk}

Labor laws impose severe penalties for working hours, rest periods, and work intensity; manual routing can lead to exceeding limits under actual time and traffic conditions, even if routes appear feasible at first. Simulation studies on labor-constrained routing also show that violations become more frequent when these limits are treated as “targets” rather than “hard constraints.” This risk is mitigated by embedding strict feasibility constraints into the optimization model of labor regulations: Although UPS drivers' contractual daily limit is approximately 9.5 hours, the model enforces a stricter fixed operational limit of 8.5 hours; routes exceeding this limit are mathematically deemed unfeasible and cannot be found, thus transforming the post-operation compliance issue into an pre-operation compliance mechanism.

\subsubsection{Algorithmic Compliance by Design}

A key legal benefit of the proposed framework is the “algorithmic compliance” approach: rather than attempting to resolve compliance issues at management discretion after routes are determined, the algorithm itself becomes part of the compliance system. This makes the company less vulnerable to labor law complaints, audits, and overtime disputes. In contrast, in manual routing, deviations caused by regulations can be covered up “in the field” with unofficial and unpaid overtime, creating a hidden accumulation of legal risk.

\subsubsection{Data Protection and Regulatory Accountability}

Apart from labor regulations, routing algorithms involve confidential business data like customer geolocations and delivery trends. According to KVKK and GDPR standards, this data has to be processed under rigorous access controls and purpose limitation. This study respects these laws by limiting access to this data to only those people who need it, while its sole use would be for research.

\subsubsection{Relevant UNSDGs}

\begin{itemize}
    \item \textbf{SDG 8 -- Decent Work and Economic Growth:}  
    Ensuring compliance with labor regulations by enforcing legally binding working-hour and workload constraints directly within the routing algorithm.
    
    \item \textbf{SDG 16.6 -- Peace, Justice and Strong Institutions:}  
    Developing effective, accountable, and transparent institutional processes by implementing rule-based, auditable, and consistently enforceable routing decisions. The optimization framework functions as a governance mechanism that reduces discretionary decision-making and strengthens organizational compliance.

\end{itemize}

\subsection{Economical Issues and Constraints}
\subsubsection{Cost structure and budget constraints}

The economic reality of last-mile operations is dominated by variable operating costs OPEX and avoidable inefficiencies. The main cost drivers are: (i) total distance and travel time (fuel, maintenance, depreciation), (ii) labor cost and overtime exposure, (iii) fleet activation and rental costs when additional vehicles are used, and (iv) operational failure costs such as re-routing, missed service performance, and dispatcher rework.

In this project, economic constraints are operationally enforced through vehicle activation penalties, distance/time penalties, and feasibility constraints that prevent infeasible plans from being ``fixed'' via overtime. Sustainable VRP literature shows that treating cost as the only objective can damage service equity and worker welfare; therefore, economic efficiency must be balanced with social feasibility constraints rather than optimized in isolation~\cite{Gouraji2025,Pilati2024}.

A second, often ignored, budget constraint is the planning window constraint: the solution must be produced before the morning shift begins. Industrial case studies identify computational tractability and operational deployability as binding economic constraints, not technical preferences~\cite{HalvorsenWeare2014}. This creates an implicit budget on compute and decision latency.

\subsubsection{ROI and business case logic}

The framework should be evaluated with a clear ROI model that separates implementation cost from operational savings.

\textbf{Typical cost items (investment + recurring):}
\begin{itemize}[leftmargin=1.5cm]
    \item \textbf{CAPEX-like one-time costs:} development, integration, data pipeline setup, training, initial parameter calibration.
    \item \textbf{OPEX recurring costs:} solver/runtime infrastructure, software licensing (if applicable), API costs for mapping/geocoding, maintenance and continuous improvement.
\end{itemize}

\textbf{Typical savings (measurable):}
\begin{itemize}[leftmargin=1.5cm]
    \item reduced distance and travel time $\rightarrow$ fuel, maintenance, and depreciation savings;
    \item reduced overtime and better shift compliance $\rightarrow$ lower labor cost volatility;
    \item fewer ``extra'' vehicles activated (or fewer rentals) $\rightarrow$ direct fleet cost savings;
    \item less dispatcher rework and fewer route failures $\rightarrow$ reduced internal failure cost.
\end{itemize}

Multi-objective sustainable routing studies show that mileage reduction can simultaneously lower operational cost and environmental burden when emissions and cost are treated as coupled outcomes~\cite{Pilati2024,Shahin2025}. Multi-period collaboration studies suggest additional savings through coordinated capacity usage across periods~\cite{Wang2020}.


\subsubsection{Financial risk and sensitivity}

ROI can collapse if (i) data quality issues inflate exception handling, (ii) compute time exceeds the morning window and forces manual fallback~\cite{HalvorsenWeare2014}, (iii) operational constraints tighten and require re-calibration, or (iv) efficiency gains induce higher demand, increasing total system load (rebound risk)~\cite{Shahin2025}. 

Rebound risk means that increased efficiency can reduce costs and simplify operations, thereby increasing demand/operational volume and potentially offsetting part or all of the expected total savings (fuel, emissions, traffic load, etc.).Therefore, the economic evaluation should include sensitivity analysis over fuel price, wage/overtime, demand surges, and solver runtime variance.

\subsubsection{Relevant UNSDGs}

\begin{itemize}[leftmargin=1.5cm]
    \item \textbf{SDG 8.2 -- Decent Work and Economic Growth:} Improving economic productivity through technological upgrading. Systematic reductions in distance/time and rework improve productivity while avoiding informal overtime~\cite{Pilati2024,HalvorsenWeare2014}.
    \item \textbf{SDG 9.4 -- Industry, Innovation and Infrastructure:} Upgrading industrial processes to increase resource-use efficiency via digital optimization and resilient planning structures~\cite{Wang2020,UN2015}.
    \item \textbf{SDG 12.2 -- Responsible Consumption and Production:} Efficient use of natural resources by reducing vehicle-kilometers traveled and increasing vehicle utilization; integrated pickup--delivery structures support resource efficiency and reverse-flow integration~\cite{Shi2023,Shahin2025}.
\end{itemize}

\subsection{Sustainability Issues}
In this project sustainability doesn't only mean "less kilometers" The goal is to make sure that operations continue to work everyday, to avoid creating unnecessary burdens on the city and the people, and to reduce resource consumption. 

\subsubsection{Environmental Sustainability}
\begin{itemize}
    \item \textbf{Local Air Quality:} Air polluters like NOx and particles, they create direct health and life quality effects in the cities. When VKT decreases, exposure decreases as well.
    \item \textbf{Noise and traffic pressure:} Better clustered routes can reduce repeated trips around the same neighborhood.
    \item \textbf{Reverse Flow(returns/pickups):} Pickup-delivery integration reduces resource usage by integrating flows such as returns/recycling into the same route instead of transporting them on a separate route.
\end{itemize}

\subsubsection{Operational Sustainability}
\begin{itemize}
    \item The rolling horizon approach aims to update the system without “crashing” it when demand changes.
    \item Route stability is critical for sustainability like if the driver's plan changes completely with each new piece of information, it will be rejected in the field.
    \item If the calculation time exceeds the time window, the system becomes unusable, which also ends sustainability.
\end{itemize}

\subsubsection{Social Sustainability}
\begin{itemize}
    \item Overly compressed routes and schedules that leave no buffer time put pressure on drivers.
In the short term, productivity increases, but in the long term, the risk of accidents/errors/resignations
increases. Therefore, a sustainable route plan must take “human capacity” constraints seriously, such as upper limits on working hours, number of stops, and realistic service times.
\end{itemize}

\subsubsection{Relevant UNSDGs}
\begin{itemize}
    \item \textbf{SDG 9.4:} Digital optimization is an operational improvement that aims to do more work with fewer resources. Digital optimization is an operational improvement that aims to do more work with fewer resources.\cite{UN2015}
    \item \textbf{SDG 11.2:} Safe, accessible, and sustainable transportation systems. Less unnecessary vehicle circulation and more predictable freight movement reduce the strain on urban mobility.\cite{UN2015}
    \item \textbf{SDG 12.2:} Sustainable management and efficient use of natural resources. Fuel, vehicle wear, and time waste are reduced; pickup–delivery integration increases resource efficiency.\cite{UN2015}
    \item \textbf{SDG 12.6:} Encourage companies to adopt and report on sustainable practices. The model is suitable for generating measurable KPIs (VKT, load factor, route stability, overtime, exception rate).\cite{UN2015}
    \item \textbf{SDG 13.2:} Integrate climate measures into policy/strategy and planning. Establishes a decision-making mechanism that reduces emissions at the planning level.\cite{UN2015}
\end{itemize}


\subsection{Producibility-Manufacturability Issues}
This project does not deliver a physical product; it delivers a decision-support service that must run reliably every day. Therefore, ``manufacturability'' should be interpreted as service deliverability and operational deployability, rather than physical producibility.

\subsubsection{Producibility as timely solution generation}

In real operations, a routing solution that is not produced before the shift starts is effectively unusable. So the producibility in this context means:
\begin{itemize}
    \item producing routes within the morning planning window,
    \item keeping runtime predictable and controllable,
    \item generating acceptable-quality routes even for large instances.
\end{itemize}

\subsubsection{Manufacturability as deployability and sustainable operation}

For a service, the main challenge is not whether it runs once, but whether it can run \textbf{consistently} every day. This requires:
\begin{itemize}
    \item robust integration with operational data pipelines (input formats, geocoding, stop attributes, capacity measures),
    \item fault tolerance and fallback behavior when data are missing or inconsistent,
    \item logging and traceability to explain why a route was generated,
    \item maintainability and version control as demand patterns and constraints evolve.
\end{itemize}

\subsubsection{ SLA and operational trust}

If the service is slow or produces inconsistent outputs, dispatchers will revert to manual planning, which effectively ends adoption. Therefore, the service must commit to practical reliability targets such as bounded runtime, minimum quality thresholds, stability under rolling updates, and minimal manual intervention.

\subsubsection{Relevant UNSDGs}

\begin{itemize}
    \item \textbf{SDG 9.5:} Enhancing technological capability by translating optimization research into a deployable industrial decision-support service.
    \item \textbf{SDG 9.4:} Upgrading operational processes through digital optimization to improve resource-use efficiency and system resilience.
\end{itemize}

\subsection{Social and Political Issues}
In this project, route planning offers the opportunity to make these conditions somewhat more manageable. For example, concentrating routes as much as possible during daytime hours, keeping load and stop density at reasonable levels, and producing predictable plans that provide clearer answers to questions like “What time will it finish today?” lowers the entry barrier for female drivers. Phung and Olapiriyakul's approach also demonstrates that route design can take into account not only cost and distance, but also social variables such as safety and predictability.~\cite{Phung2022}.


Route decisions are not solely about cost. They directly impact people's daily lives. Optimization focused solely on reducing costs naturally favors densely populated and easily accessible areas. This can lead to remote, scattered, or low-income neighborhoods receiving services later, and sometimes being systematically marginalized.~\cite{Bonku2024}.

\paragraph{Relevant UNSDGs}
\begin{itemize}
    \item \textbf{SDG 10.2:} Strengthen the social and economic inclusion of all by 2030~\cite{UN2015}.
    \item \textbf{SDG 10.3:} Reducing inequality in outcomes and ensuring equal opportunity~\cite{UN2015}.
\end{itemize}

\subsubsection{Gender Equality}
We need to be honest here. Such arrangements can sometimes take a small toll on efficiency. Choosing safer hours, easier loads, and a more balanced pace over a shorter route is not always the cheapest option. But this ensures that the system works not only for the company but also for the employee. The goal is not to give women special privileges; it is to make the job more accessible and sustainable for everyone.

There is another limitation: we cannot solve SDG 5 with algorithms alone. Route optimization can improve conditions, but it is not sufficient on its own. Without corporate security procedures, training, equipment support, and implementation discipline, it will be difficult to achieve this goal~\cite{Phung2022}.

\paragraph{Relevant UNSDGs}
\begin{itemize}
    \item \textbf{SDG 5.5 (Primary):} Ensuring women's full and effective participation and equal opportunities at all levels, including in decision-making~\cite{UN2015}.
    \item \textbf{SDG 5.2 (Secondary):} Eliminating all forms of violence against women; design decisions that reduce security risks are consistent with this goal~\cite{UN2015}.
\end{itemize}

\subsection{Environmental Issues}
Route optimization has a direct impact on environmental performance because total vehicle-kilometers traveled (VKT) is strongly linked to fuel consumption and tailpipe emissions. By reducing unnecessary detours, repeated traversals, and inefficient stop sequences, optimized routes can lower emissions and local pollutants, particularly in dense urban areas. In addition, improved planning can reduce idling time and wasted engine-on time during delays, which further decreases emissions.

However, the environmental benefit is not guaranteed automatically. Real-world emissions depend on factors that are often abstracted away in routing models, such as vehicle type (diesel vs.\ electric), load-dependent fuel burn, road gradients, and time-dependent congestion. Therefore, our results can be interpreted as an environmentally favorable direction mainly through reduced travel distance and time, while a more accurate sustainability assessment would require integrating an emission model (e.g., speed-profile or fuel-rate based) and fleet characteristics.

\paragraph{Relevant UNSDGs}
\begin{itemize}
    \item \textbf{SDG 11.6:} In line with the goal of reducing per capita environmental impact in cities, route optimization supports reducing VKT~\cite{UN2015}.
    \item \textbf{SDG 13.2:} In line with the goal of integrating climate-related measures into policy and planning, routing decisions that reduce distance/time during the planning phase provide a concrete operational mechanism for reducing CO\(_2\) emissions ~\cite{UN2015}.
\end{itemize}

\subsection{Multidisciplinary Collaboration}
This project required multidisciplinary collaboration because it sits at the intersection of operations research, data engineering, and practical logistics constraints. The routing model and constraints (capacity, stop limits, route length, feasibility) are rooted in optimization theory, while the dataset preparation demanded systematic data cleaning, preprocessing, and feature standardization (e.g., consistent stop identifiers, service types, and units). Moreover, operational realism required interpreting constraints in a way that reflects how a logistics system actually runs, not just how it looks mathematically.

Coordination among team members was essential to keep the pipeline consistent end-to-end: defining problem scope, preparing inputs, validating outputs, generating KPIs/figures, and presenting results in a report-ready format. The final workflow is therefore not only an optimization solution, but also a reproducible process that aligns modeling assumptions, computational outputs, and reporting needs across different skill areas.


% =========================
% 7. Conclusions
% =========================
\newpage
\section{CONCLUSIONS}
This project introduced a decision support framework that generates daily, actionable route plans for UPS-like last-mile pickup–delivery operations. The framework's key contribution was not merely to write a “route-generating algorithm,” but to establish a repeatable planning pipeline that integrates the following steps into a single workflow: cleaning company data → creating daily instances → generating distance/time matrices with actual road metrics → generating route candidates → validating with KPIs. This allowed modeling assumptions, data reality, and operational constraints to converge in one place.

When working with field data, the most obvious fact is this: solutions that look good in theory do not work in the field without taking into account “hard” limits such as capacity/working time/stop practicality. Therefore, the study made the constraints defining the plan's feasibility the “design inputs” from the outset: (i) capacity (Q=350) and the dynamic change in load throughout the tour when pickup and delivery are combined, (ii) the number of stops per route band (for operational practicality), (iii) shift duration (travel + service times), (iv) neighborhood/district logic for spatial realism, and (v) basic feasibility rules such as each stop being served only once and the route starting and ending at the depot. This framework focused on producing plans that could be read in operational language and monitored with KPIs, rather than plans that were “short on paper but would break down in the field.”

Monday's data contains 3,798 stops, and this scale demonstrates that the target is truly a “daily operation.” The composition of the data set is also critical: there is a workload of 3,102 deliveries, 696 pickups, and a total of 14,826 packages. These figures clarify why the method selection should be “scalable backbone + validation KPIs”: a fully optimal solution may not be practical for every instance every day; instead, a decisive, operational plan that enables rapid control is essential.

Data analysis also revealed a site-specific problem in terms of capacity: On Monday, the number of packages for 9 stops exceeded 350. These stop “load peaks” can easily be overlooked in the classic VRP assumption; however, from a real-world operational perspective, they represent exceptions that may require additional business rules such as split-service, dedicated vehicles, or pre-assignment rules. This finding demonstrates that the model must be designed to be robust not only to average stop behavior but also to extreme values.

The pilot conducted over the first 100 stops demonstrated that the framework works end-to-end and can generate KPIs: In this small instance divided into 5 clusters, direct operational metrics such as distance (km), time (min), km/stop, and min/stop were calculated for each cluster. The results clearly highlighted one point in particular:

Cluster 2 contains only 5 stops but produces a total distance of 130.19 km; with km/stop = 26.04 and min/stop = 23.04, it is by far the most “scattered” and costly route.

In contrast, dense clusters containing 17–30 stops (e.g., clusters 0, 1, 3) remain at similar levels in total distance, but their km/stop and min/stop values are much more compact.

This is a critical implication for the field: even if you balance the number of stops, the “geographic load” (travel cost) may not be balanced. Therefore, planning must directly control not only “number of stops” but also geographic spread and travel load. Thus, the anomaly produced by the existing k-means-based (density-focused) clustering in the pilot is not an error; it is evidence that enhances the study's value: The system reveals a real constraint and blind spot.

Using route metrics obtained via the Google Distance Matrix API instead of Euclidean distance made it possible to evaluate the outputs in “operational units” (km/min). This approach enhances the report's authenticity, as the decision language in operations like UPS is km and minutes.

However, this comes at a cost, and the report acknowledges this accurately: due to API quota limitations, queries must be batch processed and cached; therefore, in pilot reporting, non-traffic durations were used as a basis, while traffic-inclusive durations were additionally reported for reference. This finding shows that the system's effort to “approach reality” also collides with engineering constraints (data acquisition cost, quota management). This is a critical design issue for field deployment: instead of deploying the entire matrix with live traffic every day, a hybrid strategy (generate the plan with stable data, perform paid verification on critical lines/sample routes, and update dynamically when necessary) is more realistic.

The MIP side of the report extends the classic VRP template with UPS-specific constraints such as dynamic capacity (load increases and decreases when pickup and delivery occur simultaneously) and a stop count band. This is a methodologically correct formulation. However, an important fact that the report clearly states at this stage is that the model has not yet been computationally solved; therefore, MIP is not yet the “final optimal planner,” but rather a fundamental layer representing the decision logic the system aims for. This transparency strengthens the text because it clearly separates the output produced by the work today (heuristic-based pilot plan + KPI validation) and the next step (exact solution/optimality gap analysis with Gurobi).

This project successfully framed and piloted a decision-support approach with a scaling target based on real data from the Asian side of Istanbul. Its strongest aspect is establishing a planning line that focuses on operational constraints rather than creating an “algorithm list” and concretizes the output in a route + KPI format. Pilot analyses clearly demonstrated two key facts: Density-focused simple clustering cannot always balance geographic cost and workload (we clearly observed the few stops–long route anomaly). Field data produces outliers (such as stops with Q>350), and a realistic system must incorporate these exceptions into the model either as a rule or as separate service logic.

Therefore, this work did something far more valuable than “reinventing” the methods in the literature: it established a feasible planning backbone under the constraints, anomalies, and data acquisition costs revealed by real operational data, and made visible, with numerical evidence, which points require improvement. The improvements to be made in the next phase (constraint/assignment-based clustering, stronger cluster improvement, exact solution and gap measurement with MIP, traffic integration strategy) will be added on top of this backbone, directly increasing the system's field compatibility and quality level.


% =========================
% 8. Project Plan and Future Studies
% =========================

\clearpage
\section{PROJECT PLAN AND FUTURE STUDIES}

In the following semester, the project will evolve from heuristic exploration toward exact and metaheuristic optimization.
The next steps include:

\begin{itemize}
    \item Implementing and testing the full MIP formulation using Gurobi on clustered subproblems to achieve exact routing within each cluster.
    \item Integrating realistic traffic-adjusted travel times and dynamic load balancing.
    \item Performing comparative performance analysis between heuristic, metaheuristic, and exact solutions.
    \item Developing a user interface to visualize route outputs, KPIs, and operational alerts.
    \item Extending the model to multi-depot and time-window variants.
\end{itemize}

A final deliverable will combine the optimized model, route visualizations, and sensitivity analyses into a decision-support prototype usable by UPS operations planners.

\subsection{Rolling Horizon VRP}
The most critical design decision in rolling horizon optimization is how long decisions will be “locked” (e.g., the first X stops cannot be changed, while the rest can be revised) and the re-optimization period. DRT studies show that the performance of the rolling horizon setup (horizon length, update interval, pre-booking rate) is significantly affected  \cite{lu2023drt_rh,wangkopfer2015_rh_ctp}; similar sensitivity is expected in UPS-type distribution. Therefore, in the future, a “policy tuning” phase should be added that systematically tests different horizon/update combinations.

The current report's daily planning focus can be maintained; however, if demand and load profiles change in the operation's weekly/monthly cycles, a multi-period planning approach provides a more accurate framework \cite{athanasopoulos2013_mpvrptw,dayarian2014_branchprice}. In multi-period VRP literature, periods are represented by clustering similar production/demand levels, and the plan's “inter-period feasibility” is also managed. In our context, this means that periods such as weekdays/weekends, campaign days, or seasonal peaks are handled with separate parameters, and the plan is designed with a “deviation tolerance” for these periods.

The cost of exact methods increases rapidly on a large scale (many stops); nevertheless, exact-based methods are valuable for lower bound and quality comparisons in small/medium samples. In multi-period VRP, exact approaches such as branch-and-price have been shown to yield solvable results up to certain sizes \cite{athanasopoulos2013_mpvrptw,dayarian2014_branchprice}. Therefore, in the future, an exact (or exact-derived) “benchmark module” should be established for small samples, and the quality of the heuristic/rolling horizon solution used for large samples should be measured against this reference.

In a rolling horizon dynamic structure, not only “algorithm selection” but also minor interventions added to the problem (e.g., vehicle under-occupancy penalties, capacity reduction assumptions) can affect overall performance. In multi-period dynamic distribution networks, it has been reported that interventions such as penalties that encourage vehicle usage or capacity/utilization-based regulations can reduce total travel time and the number of vehicles \cite{harmse2025_thesis}.

In the rolling horizon + heuristic approach, generating different starting points for the same input reduces getting stuck in local minima. In VRP, random sequences, Monte Carlo-based trials, and different random number generators have been shown to affect solution quality \cite{FisherJaikumar1981,golden2008_vrp}. Therefore, in the future, controlled randomization (multiple runs, seed management) should be added to the solver module; “average performance” and “worst-case scenario” should be reported based on the obtained solution distribution. This significantly increases the methodological reliability of the report.

\subsection{Multi-period VRP}
Multi-period VRP (MPVRP) addresses routing decisions not as a single day/single plan, but across multiple periods (days/weeks/shifts). This perspective is more realistic, especially when demand and operational intensity vary across periods (weekdays–weekends, campaign days, seasonal effects). The fundamental challenge in MPVRP is not so much to optimize “that day's” route in each period, but rather to balance consistency across periods (vehicle and resource continuity, visit frequency targets, service levels) with total cost. Therefore, in the literature, MPVRP is mostly modeled with classic constraints such as time windows and capacity, along with binding decisions across periods; thus, the plan remains operationally feasible not just for a single period, but throughout the entire horizon.\cite{athanasopoulos2013_mpvrptw}

The value of MPVRP for the follow-up work of this report lies in the following: When switching to a rolling horizon structure, the plan will already be updated periodically; the MPVRP framework ensures that these updates go beyond being merely “instant improvements” and serve multi-period objectives. For example, balancing visit frequencies in specific regions, distributing vehicle workloads more stably across periods, and addressing certain customer types (such as high-volume shippers) with period-based strategies can be more clearly defined using MPVRP logic. However, this expansion honestly comes at an additional cost: the problem size and decision space grow; therefore, while exact methods in the literature are generally limited in scale, decomposition approaches such as branch-and-price or well-designed heuristic/meta-heuristic frameworks come to the fore \cite{dayarian2014_branchprice,harmse2025_thesis}. In our context, the right approach is to generate an exact/decomposition-based benchmark on small samples and report the heuristic/rolling horizon solution for large scales using MPVRP metrics (total cost + inter-period stability)\cite{dayarian2014_branchprice}.

\subsection{Fisher and Jaikumar Method For Cluster-first/Route Second Algorithm}
The Fisher-Jaikumar method designs the “cluster-first, route-second” logic to focus on capacity and feasibility. In VRP, customer-vehicle assignment (which customer goes to which vehicle) and the order of visits within a vehicle (the order in which the route is followed) are normally two intertwined decisions; this method simplifies the problem structure by first strongly solving the assignment decision. The basic idea is this: Once the assignments are fixed, for each vehicle, only a “tour” problem (in practice, a TSP-type ordering problem) remains over the customers assigned to that vehicle. Therefore, the method produces good clusters/assignments first and then derives the route order for each cluster, rather than searching for the route directly as a whole; this is a clear engineering choice aimed at increasing applicability in large examples.\cite{FisherJaikumar1981}.

The “cluster-first” part of the method is based on a generalized assignment problem, not a simple geographic clustering. First, a “seed” customer (initial representative point) is selected for each vehicle to define the depot-seed-depot skeleton. Then, each customer \(i\) to vehicle \(k\)assigned and its cost \(d_{ik}\) is calculated; This cost is structured to represent the marginal cost increase resulting from the customer's addition to the relevant seed-based skeleton. Then, \(\sum_{i}\sum_{k} d_{ik} y_{ik}\) The assignment model, which includes constraints such as purpose and capacity, is solved; thus, the resulting clusters are “capacity-respecting” from the outset and closer to the feasibility of the VRP. In the final stage, “route-second” comes into play: For each vehicle, a tour/sequencing heuristic is run on the assigned customer cluster to obtain the visit order. \cite{FisherJaikumar1981}.

For the continuation of our report, the Fisher-Jaikumar method is a strong candidate for producing a “fast and constraint-sensitive initial solution,” especially at large scales. The most realistic development path is to use this approach as an initial/solution generation module and add more aggressive improvement steps (such as intra-route relocations, swaps, local search) on top of it; thus, both the initial feasibility is preserved and the final solution quality is improved. Furthermore, the method's own structure generates clear research questions for future studies: systematizing seed selection, \(d_{ik}\) The aim is to experimentally report the effect of iteratively updating costs to better align assignment costs with actual route costs and adding operational constraints outside of capacity (e.g., route duration/working hour limits) as secondary constraints to the assignment model on performance. \cite{FisherJaikumar1981}.


\subsection{Updated Gantt Chart}
In our previous work, we obtained operational data for the Asian side of Istanbul and prepared it for analysis; we clarified the problem definition according to UPS operational constraints (capacity, stop upper/lower limits, shift duration, and neighborhood/community logic) and began to establish and advance a basic MIP model framework that fits this framework. We also created the necessary narrative structure for progress reports and end-of-period reports on the reporting side and conducted initial trials with the Google Distance API for realistic calculation of distance/time metrics. The focus in the next phase will be to solve and validate the MIP model at a pilot scale using Gurobi, then strengthen constructive and corrective heuristics within the cluster-first route-second backbone to produce applicable results when the scale increases. simultaneously, the interface design for presenting the decision support framework to the user will be completed, and all findings will be converted into a final report and final presentation/poster output.

\begin{figure}[h!]
    \centering
    \includegraphics[width=\textwidth]{Preparing Project Final Presentation or Poster.png}
    \caption{Project timeline (Gantt view).}
    \label{fig:project-timeline}
\end{figure}
\newpage
\begin{itemize}
    \item \textbf{10--21 November:} Retrieving data from UPS and cleaning it to make it ready for analysis.
    \item \textbf{20--28 November:} Initiating the mathematical modeling study and preparing the progress report during the same period (24--28 November).
    \item \textbf{1 December -- 2 January:} Model improvement/updating; Google Distance API trials during this period (27 December -- 2 January) and finishing the Final Report (31 December -- 4 January) 
    \item \textbf{3--15 January:} Final Presentation.
    \item \textbf{15 January -- 6 March:} Solving the mathematical model with Gurobi and making the necessary upgrades to the model.
    \item \textbf{20 February -- 28 April:} Developing constructive and remedial insights for large scale (strengthening the cluster-first/route-second approach).
    \item \textbf{24 April -- 26 May:} User interface design (presentation/application side of the decision support framework).
    \item \textbf{21 May -- 19 June:} Project final report writing and submission preparation.
    \item \textbf{1--19 June:} Final presentation/poster preparation and submission process (progressing in parallel with the final report).
\end{itemize}


% \begin{figure}[h]
% \centering
% \includegraphics[width=0.9\textwidth]{gantt_chart.png}
% \caption{Project plan (Gantt chart) --- replace with your updated chart}
% \label{fig:gantt}
% \end{figure}

% =========================
% 9. References
% =========================
\clearpage

%1. Y. Shi, Y. Lin, S. Wang, H. Wen, M. K. Lim, and Y. Li, “A simultaneous facility location and vehicle routing problem with recyclable express packaging consideration for sustainable city logistics,” Sustainable Cities and Society, vol. 98, p. 104857, 2023. 2. R. Ebrahimi Gouraji, H. Soleimani, and B. Afshar-Najafi, “Optimization of Sustainable Vehicle Routing Problem Taking into Account Social Utility and Employing a Strategy with Multiple Objectives,” International Journal of Engineering, Transactions A: Basics, vol. 38, no. 7, pp. 1631–1658, 2025. 3. F. Pilati and R. Tronconi, “Tri-Objective Vehicle Routing Problem to Optimize the Distribution Process of Sustainable Local E-Commerce Platforms,” Sustainability, vol. 16, no. 5, Art. 1810, Feb. 2024. 4. R. Bonku, F. Alkaabneh, and L. B. Davis, “Collaborative vehicle routing for equitable and effective food allocation in nonprofit settings,” J. Humanitarian Logistics Supply Chain Manage., vol. 15, no. 2, pp. 135–155, 2024. 5. P. T. K. Phung and S. Olapiriyakul, “Sustainable Vehicle Routing: Optimizing for Gender Equality,” in Proc. 16th Int. Congress on Logistics and SCM Systems (ICLS), Khon Kaen, Thailand, Aug. 2022, pp. 58–65. 6. A. Sadeghi, F. Aros-Vera, H. Mosadegh, and R. Younes-Sinaki, “Social cost-vehicle routing problem and its application to the delivery of water in post-disaster humanitarian logistics,” Transportation Research Part E, vol. 176, Art. 103189, 2023. 7. Y. Wang, Q. Li, X. Guan, J. Fan, Y. Liu, and H. Wang, “Collaboration and resource sharing in the multidepot multiperiod vehicle routing problem with pickups and deliveries,” Sustainability, vol. 12, no. 15, Art. 5966, 2020. 8. R. Shahin and M. A. Dulebenets, “From cost-centering to sustainability: A review of Pollution Routing Problems,” Cleaner Engineering and Technology, vol. 29, Art. 101082, 2025. 9. M. Shiri and F. Ahmadizar, “An equitable and accessible vaccine supply chain network in the epidemic outbreak of COVID-19 under uncertainty,” Journal of Ambient Intelligence and Humanized Computing, vol. 14, pp. 14695–14719, 2023. 10. S. Halvorsen-Weare and M. W. P. Savelsbergh, “Labour-Constrained and Equipment-Constrained Routing Problems: Insights from Industrial Case Studies,” in Vehicle Routing: Problems, Methods, and Applications, 2nd ed. Philadelphia, PA: SIAM, 2014, pp. 367–386. 11. United Nations (2015). Transforming our world: the 2030 Agenda for Sustainable Development. [Online]. Available: https://sdgs.un.org/2030agenda (accessed Nov. 30, 2025).
% Option A: Manual IEEE numbered list
\bibliographystyle{IEEEtran}
\bibliography{references}
% =========================
% 10. Appendix
% =========================
\clearpage

\appendix
\thispagestyle{empty}
\section{Appendix, additional tables}
\begingroup
\small
\begin{xltabular}{\textwidth}{@{} c X l r @{}}
\caption{Example route sequence for Monday cluster 0 (pilot on first 100 stops).}
\label{tab:route_seq_cluster0}\\
\toprule
\textbf{Seq} & \textbf{Stop ID} & \textbf{Service Type} & \textbf{Quantity} \\
\midrule
\endfirsthead

\toprule
\textbf{Seq} & \textbf{Stop ID} & \textbf{Service Type} & \textbf{Quantity} \\
\midrule
\endhead

\midrule
\multicolumn{4}{r}{\textit{Continued on next page}} \\
\midrule
\endfoot

\bottomrule
\endlastfoot

0 & DEPOT & DEPOT & -- \\
1 & 6 & Delivery & 1 \\
2 & 5 & Delivery & 1 \\
3 & 39 & Delivery & 1 \\
4 & 38 & Delivery & 4 \\
5 & 36 & Delivery & 1 \\
6 & 7 & Delivery & 1 \\
7 & 1 & Delivery & 1 \\
8 & 2 & Pickup & 28 \\
9 & 37 & Delivery & 1 \\
10 & 4 & Delivery & 1 \\
11 & 46 & Delivery & 1 \\
12 & 27 & Delivery & 26 \\
13 & 59 & Delivery & 1 \\
14 & 3 & Delivery & 1 \\
15 & 54 & Delivery & 1 \\
16 & 53 & Delivery & 1 \\
17 & 25 & Pickup & 81 \\
18 & DEPOT & DEPOT & -- \\
\end{xltabular}
\endgroup

\begin{table}[h]
\centering
\caption{High-volume outlier stops in Monday data (Qty $>$ 350) requiring potential split-service handling.}
\label{tab:qty_outliers}
\resizebox{0.7\textwidth}{!}{% % Tablo çok geniş olmadığı için 0.7\textwidth daha dengeli durur
    \begin{tabular}{l c r c}
    \toprule
    \textbf{Stop ID} & \textbf{Service Type} & \textbf{Quantity} & \textbf{Cluster} \\
    \midrule
    1948 & Pickup & 1034 & 27 \\
    3520 & Pickup & 673 & 3 \\
    2126 & Delivery & 672 & 1 \\
    2535 & Pickup & 596 & 3 \\
    1873 & Pickup & 568 & 3 \\
    1155 & Pickup & 554 & 22 \\
    2370 & Pickup & 393 & 3 \\
    5139 & Pickup & 388 & 3 \\
    2452 & Delivery & 387 & 3 \\
    \bottomrule
    \end{tabular}%
}
\end{table}
\begin{landscape}
\begin{table}[p]
\centering
\caption{Sample pairwise travel distance matrix (km) for depot + 10 stops (Google Distance Matrix API).}
\label{tab:sample10_dist_km_landscape}
\vspace{10pt}
\small % Yatay sayfada ferah bir görünüm sağlar
\setlength{\tabcolsep}{6pt} % Sütunlar arası boşluğu optimize ediyoruz
\begin{tabular}{l | r r r r r r r r r r r}
\toprule
\textbf{Stop} & \textbf{Depot} & \textbf{1} & \textbf{2} & \textbf{3} & \textbf{4} & \textbf{5} & \textbf{6} & \textbf{7} & \textbf{8} & \textbf{9} & \textbf{10} \\
\midrule
\textbf{Depot} & 0.00 & 9.55 & 6.06 & 8.05 & 2.73 & 4.54 & 2.32 & 7.08 & 30.98 & 64.28 & 15.82 \\
\textbf{1}     & 9.62 & 0.00 & 3.48 & 11.78 & 10.69 & 8.27 & 11.94 & 1.81 & 41.62 & 76.84 & 28.38 \\
\textbf{2}     & 6.13 & 3.49 & 0.00 & 8.30 & 7.21 & 4.79 & 8.45 & 4.57 & 38.14 & 71.45 & 27.12 \\
\textbf{3}     & 8.84 & 10.33 & 6.84 & 0.00 & 9.92 & 3.18 & 8.33 & 8.41 & 36.99 & 70.30 & 21.84 \\
\textbf{4}     & 3.99 & 10.18 & 6.70 & 11.98 & 0.00 & 8.46 & 4.35 & 11.00 & 28.41 & 58.08 & 18.37 \\
\textbf{5}     & 4.09 & 7.87 & 4.38 & 4.21 & 5.84 & 0.00 & 5.43 & 5.95 & 34.09 & 67.40 & 18.94 \\
\textbf{6}     & 3.33 & 10.35 & 6.86 & 7.09 & 5.09 & 2.87 & 0.00 & 10.41 & 33.34 & 66.65 & 18.18 \\
\textbf{7}     & 8.87 & 1.77 & 4.57 & 7.94 & 9.94 & 6.21 & 11.18 & 0.00 & 38.05 & 71.36 & 22.90 \\
\textbf{8}     & 28.68 & 43.86 & 40.31 & 38.90 & 27.63 & 33.61 & 31.13 & 40.47 & 0.00 & 32.88 & 16.02 \\
\textbf{9}     & 58.48 & 54.93 & 62.90 & 68.71 & 57.43 & 63.41 & 60.94 & 70.28 & 33.30 & 0.00 & 45.83 \\
\textbf{10}    & 19.40 & 32.17 & 28.62 & 27.21 & 18.87 & 21.91 & 19.43 & 28.77 & 16.38 & 46.04 & 0.00 \\
\bottomrule
\end{tabular}
\end{table}
\end{landscape}

\input{assets/tables/tablsample10durationtrafficmin.tex}
\begin{landscape}
\begin{table}[p]
\centering
\caption{Sample pairwise travel duration-in-traffic matrix (min) for depot + 10 stops (Google Distance Matrix API).}
\label{tab:sample10_dur_min}
\vspace{10pt}
\small
\setlength{\tabcolsep}{6pt}
\begin{tabular}{l | rrrrrrrrrrr}
\toprule
\textbf{Stop} & \textbf{DEPOT} & \textbf{1} & \textbf{2} & \textbf{3} & \textbf{4} & \textbf{5} & \textbf{6} & \textbf{7} & \textbf{8} & \textbf{9} & \textbf{10} \\
\midrule
\textbf{DEPOT} & 0.38 & 13.50 & 8.28 & 10.90 & 4.10 & 5.83 & 3.18 & 8.57 & 27.20 & 49.02 & 19.23 \\
\textbf{1}     & 13.23 & 0.00 & 5.33 & 17.83 & 15.25 & 12.73 & 16.42 & 4.15 & 38.08 & 59.73 & 30.05 \\
\textbf{2}     & 7.92 & 5.23 & 0.15 & 12.50 & 9.95 & 7.42 & 11.12 & 7.70 & 32.67 & 54.47 & 24.83 \\
\textbf{3}     & 11.37 & 15.92 & 10.70 & 0.00 & 13.38 & 5.12 & 12.07 & 12.00 & 36.03 & 57.82 & 28.13 \\
\textbf{4}     & 5.47 & 15.38 & 10.18 & 16.27 & 0.00 & 11.17 & 5.63 & 13.92 & 26.98 & 48.97 & 19.78 \\
\textbf{5}     & 6.77 & 13.40 & 8.18 & 7.42 & 8.88 & 0.17 & 7.97 & 9.48 & 31.93 & 53.75 & 24.02 \\
\textbf{6}     & 5.15 & 17.73 & 12.53 & 11.62 & 7.25 & 5.43 & 0.00 & 13.75 & 30.32 & 52.13 & 22.38 \\
\textbf{7}     & 10.68 & 3.98 & 7.78 & 14.55 & 12.70 & 9.13 & 13.87 & 0.00 & 36.35 & 58.15 & 28.43 \\
\textbf{8}     & 29.13 & 41.08 & 36.48 & 37.83 & 28.23 & 32.98 & 28.77 & 36.17 & 0.00 & 31.37 & 21.35 \\
\textbf{9}     & 52.45 & 66.13 & 59.02 & 61.28 & 51.57 & 56.52 & 52.38 & 59.78 & 33.67 & 0.00 & 44.72 \\
\textbf{10}    & 19.25 & 30.93 & 26.28 & 27.72 & 18.98 & 22.85 & 18.60 & 26.00 & 20.13 & 42.18 & 0.00 \\
\bottomrule
\end{tabular}
\end{table}
\end{landscape}
\section{Appendix , additional figures}

\begin{figure}[h]
\centering
\includegraphics[width=0.85\textwidth]{assets/figs/kpi_total_km_first100.png}
\caption{Total distance (km) by cluster for the first 100 stops.}
\label{fig:kpi_km_first100}
\end{figure}

\begin{figure}[h]
\centering
\includegraphics[width=0.85\textwidth]{assets/figs/kpi_total_min_first100.png}
\caption{Total duration (minutes) by cluster for the first 100 stops.}
\label{fig:kpi_min_first100}
\end{figure}

\end{document}